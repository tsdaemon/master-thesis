\chapter{Introduction}
\label{Chapter1}

\newcommand{\keyword}[1]{\textbf{#1}}
\newcommand{\tabhead}[1]{\textbf{#1}}

\section{Motivation}
Software development is often described as a knowledge-intensive field \parencite{Robillard1999}. Implementation and maintenance of enterprise software systems require broad knowledge of various programming languages and application of programming interfaces. While in 2002 to create a website a developer had to know HTML/CSS, PHP and MySQL, in 2017 this requires knowledge about frontend ecosystem, backend frameworks, and different NoSQL query languages. Documentation becomes a bottleneck while solving simple tasks, especially for new developers. For that reason software development involves regular use of search engines and Q\&A databases \parencite{Treude2011}. Code snippets from crowdsourced resources like StackOverflow are adopted and reused in other projects. Developers often seek to find existing examples of working code to solve regular tasks instead of writing and testing it from scratch \parencite{Brandt2010}. And to find corresponding code snippet, the software developers first formulate its description as a query for search engine \parencite{Brandt2009}. 

However, web search is a time-consuming task, which causes interruptions of the coding process. As an alternative, the code description could be translated directly to code. Such translation tool would reduce the burden of remembering the details of a particular language or API and allow a developer to use his or her time for more creative aspects of development. That was a motivation of the present work: we wanted to develop a model of description-to-code translation, which could transform informal instrcutions to actual language specific implementation.

%to create dynamic, interactive and easy-to-use code generation tool which would allow to translate code description to actual implementation. 

% It uses a code description to generate a required snippet of code just under the programmer cursor.

%----------------------------------------------------------------------------------------

\section{Goals of the master thesis}

\begin{enumerate}
	\item To explore previous examples of code generation or code search tools.
	\item To train Description2Code syntactic model and compare its performance to previous related works.
	\item To develop code generation plugin for PyCharm IDE.
\end{enumerate}

\section{Thesis structure}
This work is structured as follows: In Chapter \ref{Chapter2} we presented references to related publications and comparison with previous code generation projects. In Chapter \ref{Chapter3} we provided theoretical background for methods used in this work. In Chapter \ref{Chapter4} we explained the idea of Tree2Tree model with all details about its structure and implementation. In Chapter \ref{Chapter5} we described data preprocessing, explained model implementation details and presented evaluation results. And finally, in Chapter \ref{Chapter6} we drawn the conclusion and set the points for further research.