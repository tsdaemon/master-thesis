\chapter{Python 3.6 abstract syntax tree grammar} 
\label{AppendixA} 

The following grammar is taken from official documentation of Python module \href{https://docs.python.org/3/library/ast.html}{ast}. It contains description of all AST rules. Each rule describes how specific node type expands to other nodes and what types of children it could contain. Modifier \code{*} denotes that there could be multiple child nodes of that type. Modifier \code{?} denotes that this child node is optional. For example, \code{Assert(expr test, expr? msg)} mean that node \code{Assert} requires expression to test and optionally it could contain expression with message.

\begin{verbatim}
-- ASDL's 7 builtin types are:
-- identifier, int, string, bytes, object, singleton, constant
--
-- singleton: None, True or False
-- constant can be None, whereas None means "no value" for object.

module Python
{
    mod = Module(stmt* body)
        | Interactive(stmt* body)
        | Expression(expr body)

        -- not really an actual node but useful in Jython's typesystem.
        | Suite(stmt* body)

    stmt = FunctionDef(identifier name, arguments args,
                       stmt* body, expr* decorator_list, expr? returns)
          | AsyncFunctionDef(identifier name, arguments args,
                             stmt* body, expr* decorator_list, expr? returns)

          | ClassDef(identifier name,
             expr* bases,
             keyword* keywords,
             stmt* body,
             expr* decorator_list)
          | Return(expr? value)

          | Delete(expr* targets)
          | Assign(expr* targets, expr value)
          | AugAssign(expr target, operator op, expr value)
          -- 'simple' indicates that we annotate simple name without parens
          | AnnAssign(expr target, expr annotation, expr? value, int simple)

          -- use 'orelse' because else is a keyword in target languages
          | For(expr target, expr iter, stmt* body, stmt* orelse)
          | AsyncFor(expr target, expr iter, stmt* body, stmt* orelse)
          | While(expr test, stmt* body, stmt* orelse)
          | If(expr test, stmt* body, stmt* orelse)
          | With(withitem* items, stmt* body)
          | AsyncWith(withitem* items, stmt* body)

          | Raise(expr? exc, expr? cause)
          | Try(stmt* body, excepthandler* handlers, stmt* orelse, stmt* finalbody)
          | Assert(expr test, expr? msg)

          | Import(alias* names)
          | ImportFrom(identifier? module, alias* names, int? level)

          | Global(identifier* names)
          | Nonlocal(identifier* names)
          | Expr(expr value)
          | Pass | Break | Continue

          -- XXX Jython will be different
          -- col_offset is the byte offset in the utf8 string the parser uses
          attributes (int lineno, int col_offset)

          -- BoolOp() can use left & right?
    expr = BoolOp(boolop op, expr* values)
         | BinOp(expr left, operator op, expr right)
         | UnaryOp(unaryop op, expr operand)
         | Lambda(arguments args, expr body)
         | IfExp(expr test, expr body, expr orelse)
         | Dict(expr* keys, expr* values)
         | Set(expr* elts)
         | ListComp(expr elt, comprehension* generators)
         | SetComp(expr elt, comprehension* generators)
         | DictComp(expr key, expr value, comprehension* generators)
         | GeneratorExp(expr elt, comprehension* generators)
         -- the grammar constrains where yield expressions can occur
         | Await(expr value)
         | Yield(expr? value)
         | YieldFrom(expr value)
         -- need sequences for compare to distinguish between
         -- x < 4 < 3 and (x < 4) < 3
         | Compare(expr left, cmpop* ops, expr* comparators)
         | Call(expr func, expr* args, keyword* keywords)
         | Num(object n) -- a number as a PyObject.
         | Str(string s) -- need to specify raw, unicode, etc?
         | FormattedValue(expr value, int? conversion, expr? format_spec)
         | JoinedStr(expr* values)
         | Bytes(bytes s)
         | NameConstant(singleton value)
         | Ellipsis
         | Constant(constant value)

         -- the following expression can appear in assignment context
         | Attribute(expr value, identifier attr, expr_context ctx)
         | Subscript(expr value, slice slice, expr_context ctx)
         | Starred(expr value, expr_context ctx)
         | Name(identifier id, expr_context ctx)
         | List(expr* elts, expr_context ctx)
         | Tuple(expr* elts, expr_context ctx)

          -- col_offset is the byte offset in the utf8 string the parser uses
          attributes (int lineno, int col_offset)

    expr_context = Load | Store | Del | AugLoad | AugStore | Param

    slice = Slice(expr? lower, expr? upper, expr? step)
          | ExtSlice(slice* dims)
          | Index(expr value)

    boolop = And | Or

    operator = Add | Sub | Mult | MatMult | Div | Mod | Pow | LShift
                 | RShift | BitOr | BitXor | BitAnd | FloorDiv

    unaryop = Invert | Not | UAdd | USub

    cmpop = Eq | NotEq | Lt | LtE | Gt | GtE | Is | IsNot | In | NotIn

    comprehension = (expr target, expr iter, expr* ifs, int is_async)

    excepthandler = ExceptHandler(expr? type, identifier? name, stmt* body)
                    attributes (int lineno, int col_offset)

    arguments = (arg* args, arg? vararg, arg* kwonlyargs, expr* kw_defaults,
                 arg? kwarg, expr* defaults)

    arg = (identifier arg, expr? annotation)
           attributes (int lineno, int col_offset)

    -- keyword arguments supplied to call (NULL identifier for **kwargs)
    keyword = (identifier? arg, expr value)

    -- import name with optional 'as' alias.
    alias = (identifier name, identifier? asname)

    withitem = (expr context_expr, expr? optional_vars)
}

\end{verbatim}